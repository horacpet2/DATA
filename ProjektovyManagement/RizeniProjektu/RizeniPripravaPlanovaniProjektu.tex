%hlavička dokuentu

\input Konfigurace
\input TitulniStrana

%nastavení posunutí počítání stránek (úvodní stráka se nepočítá do celkového
% počtu stránek)
\pageno = 1

%vytvoření automaticky generovaného obsahu dokumentu
\nonum\notoc\sec Obsah

\maketoc

\vfill\break

%spuštění počítání stránek
\pagenumbers

%tělo dokumentu

\chap Úvod

Člověk je ve své povaze velice líný a zvědavý tvor. Díky tomu jsme ale jako civilizace tam kde teď jsme. Máme neustále potřebu něco zkoumat a vylepšovat, abychom si tím ušetřili čas a zjednodušili práci. To byl hlavní popud, který nás vedl ve vývoji od používání kamenných nástrojů v pravěku až k dnešní moderní technice. Ve vývoji lidské civilizace se ale neustále zvětšuje objem používaných znalostí, které slouží ke zdokonalování a urychlení nespočtu činností a samotného vývoje. Vyráběné nástroje, stroje a technyky jsou stále složitější jejich vývoj dražší. Z tohoto důvodu bylo potřeba v určitém bodě začít nahlížet na jejich vývoj a realizaci trochu jinak. Bylo potřeba vytvořit sadu technik, které slouží jako univerzální postup, podle kterého se postupuje při vývoji nových produktů, aby se efektivně využily dostupné zdroje.

Díky tomu se z obyčejného nákresu na kousku papíru a pár poznámkách na okraji stal nový obor lidské činnosti, který mají na staroti specializovaní odborníci. Tímto oborem se stal {\bf projektový management}. Projektový manažéři se starají, aby se při návrhu a realizaci nového produktu efektivně využilo dostupných prostředků, které jsou rozděleny mezi jednotlivé projektové týmy a aby vývoj mezi projektovými tými probíhal synchronně - spolupráce více týmů na jednom úkolu.

Díky projektovému managementu se vývoj nového produktu výrazně ury\-chlí a zamezí se zbytečnému plýtvání zdrojů. Celý projekt je totiž rozdělen na několik souvisejících částí, a přiděleny různým projektovým týmům. Projektový manažér pak s celkovým přehledem nad dostupnými zdroji tyto kontrolovaně rozděluje mezi jednotlivé tými tak, aby se zamezilo plýtvání.


\chap Projekt

Při přípravě, plánování a realizaci projektu je třeba nejprve definovat co to projekt vlastně je a co je jejím cílem. Výsledkem libovolného projektu je nějaký produkt. Produkt je něco čím má být obohacen život lidstva, nebo jen skupiny lidí, nebo dokonce jen určitých jednotlivců, nebo tím má být urychlen, zjednodušen, zefektivněn vývoj nějakých jiných technolovií. Výsledkem projektu by tedy mělo být něco přesně definovaného, co plní nějaký přesně (předem) definovaný účel.

Projekt a jeho realzace je tedy pouhým prostředkem k vytvoření daného produktu (služby). Konečným cílem všech myšlenkových úvah není projekt, ale jeho konečný produkt. Produkt může být velice různorodý. Může mít povahu nějakého předmětu, zařízení, služby, poznatku, ... Úvahy o projetu tedy začínají úvahamy o produktu. Charakter produktu pak určuje průběh projektu. Projekt je možné definovat jako soustavu aktivit (činností) směřujících k předem stanovenému cíli, která má určitý začátek i konec. Vyžaduje spolupráci různorodých činností, profesí a využívá je k vytvoření výstupu - produktu.

Každý projekt je vždy spojen s rizikem neuspěchu, protože každý projekt je jedinečný a nikdy nelze předem předvídat co v jeho průběhu nastane. Z toho vyplývá, že každý projekt musí být flexibilní, aby mohl vhodně reagovat na nečekané situace. Možnost neúspěchu nelze vždy chápat jako negativní výsledek. I neúspěch projektu může být pozitivní v tom, že zvolený způsob, cesta, řešení nebo obecně každý zvolený postup tak může být poučný do budoucna a to ne jen pro samotného řešitele, ale i pro ostatní. Je důležité se z každého neúspěchu nějak poučit a využít z neúspěchu maximum, aby se daná chyba v budoucnu již neopakovala. 

\sec Definice projektu

Projekt může být definován jako sled nutných úkonů, které jsou potřeba vykonat k získání žádaného výstupů - produktu.

Počátečním impulzem pro zahájení projektu je potřeba uspokojit nějakou potřebu, něco inovovat, něco vyprodukovat, ... Proto před zahájením projektu je třeba si položit otázku "proč". Co je motivem pro zahájení projektu a proč by se měl uzkutečnit. Libovolná potřeba, která je impulzem pro zahájení projektu se projevuje jako vzniklý problém v probíhajícím procesu, jako nedostatečná odpověď na kladenou otázku. Například nevyhovující technologie, která neumožňuje vhodným způsobem řešit určité úkony. Produktem projektu by tedy mělo být řešení daného problému. 

Postup přípravy projektu má tedy podobu několika bodů:

\begitems
*Potřeba
*Možnosti uspokojení potřeb
*Prozkoumání možností
*Výběr vhodného řešení
*Definování projektu
\enditems

Možnosti uspokojení vědomé potřeby jsou první myšlenky o tom co je třeba vykonat pro to, aby byla potřeba uspokojena. Jsou to tedy první nápady, týkající se možného řešení, které se nabýzejí pro uspokojování potřeb. Tyto myšlenky umožní již v počátcích zjisti, zda je projekt v daných podmínkách v dohledné době realizovatelný. Následkem toho je výběr vhodného řešení, které je v souladu s definovanou potřebou a možnostmi. Jako poslední je definice projektu ve které je projektový záměr transformován do konkrétní podoby. V tomto bodě se definují cíle, strategie, zdroje, postupy, ...

\sec Životní cyklus projektu

I když je každý projekt unikátní, jeho průběh by měl být vždy více méně stejný. Žívotní cyklus projektu se dá rozdělit do čtyř samostatných fází:

\begitems
*Koncepce - zjišťování požadavků, vypracování alternativ, odhad zdrojů
*Plánování (vývoj) - sestavení základního týmu, zpracování plánu projetu, start projektu
*Realizace - přímá realizace projektu, jeho sledování a řízení
*Ukončení - kontrola a přijetí projektu, dokumentace a zhodnocení výsledků
\enditems

Celý proces začíná zákldní úvahou, formulování myšlenky o tom, čeho má být dosaženo, pokračuje definováním aktivit a končí reálným dosažením stanovených cílů. Životní cyklus projektu představuje přesně formulované fáze projektu, plán všech činností spojené s projektem. Jedině tak lze zajistit logičnost celého postupu a vyhnout se tak nepředvídatelným situacím, které mohou narušit postup projektu (se kterými je ale přes sebelepší plát třeba počítat). V každé fázi projektu je třeba porovnat reálný postup projektu s plánovaným a tak kontrolovat jeho postup.

Celý projekt končné ve chvíli kdy jsou cíle a úkoly naplněny a uplynula doba určená k realizaci projektu. Byl tedy vytvořen konečný produkt. Při ukončování projektu je třeba vytvořit závěrečnou zprávu, která je součástí dokumentace. Se závěrečnou zprávou souvisí také audit finančních nákladů spojených s realizací projektu.

\sec Projektová hierarchie

Na projektu se můžou podílet jednotlivci, jedna nebo více spolupracujících týmových skupin, nebo dokonce celé organizace. V případě, že se na projektu podílí více lidí, je třeba, aby v projektu vznikla nějaká hierarchie autorit, které mají přesně rozdělené své pravomocé povinosti. Tím je zajištěno výrazné zrychlení průběhu celého životního cyklu projektu.

Nejvýše v hierarchii je umístěn {\bf projektový vedoucí}. Ten má na starosti vypracování projektové dokumentace a plánování technické části projektu. Projektovému vedoucímu se také říká {\bf řešitel}, protože musí řešit ne jen problémy realizace projektu, ale i případné problémy, které nastanou při jeho realizaci. 

Po projektovém vedoucím se nachází {\bf projektový manager}, který má na starosti získávání a rozdělování dostupných zdrojů mezi realizační tými. Projektový manager musí neustále komunikovat se členy týmu a projektovým vedoucím, aby měl aktuální přehled o stavu projektu a na jeho základě mohl poskytovat zdroje. 

Jako další je v hierarchii {\bf týmový vedoucí}, který přijímá úkoly od řešitele a managera a zadává pkoly jednotlivým členům svého týmu. Na jednom projektu se může podílet více týmů, podle složitosti projektu a jeho požadavků na odbornou způsobilost. 

V každém projektovém týmu by měl být jeden {\bf zástupce týmového vedoucího}, který by v případě potřeby dokázal týmového vedoucího dočasně zastoupit. Tato funkce může být jakási příprava na budoucí postup v hierarchii na pozici týmového vedoucího. 

Jako poslední se v projektové hierarchii nacházejí jednotlivý {\bf členové projektového týmu}. Ti mají za úkol ralizaci zadaných prací.

Mezi jednotlivými členy projektové hierarchie musí probíhat neustálá komunikace, aby každý z jeho členů měl přehled o aktuálním stavu projektu a v případě nastalého problému se mohla učinit včasná opatření.

\chap Projektová dokumentace

Projektová, nebo v případě technicky zaměřených projektů také technická dokumentace je textová, obrázková, popřípadě i jiným průvodní záznam (popis) průběhu realizace složitých projektů (stavby, stroje, ...). Dokumentace projektu umožňuje popsat jak daný projekt vznikal, jak jej pro příště optimalizovat, jak naplánovat jeho realizaci (charakteristiky projektu) a jak jeho realizaci v budoucnu zopakovat, nebo čemu se vyhnout. Jedná se o velice důležitou část každého složitého projektu, protože umožňuje udržet přehlednost projektu a zamezuje zbrklému improvizování, které vede k tvorbě zbytečných chyb a šetří prostředky potřebné k realizaci - poskytuje informace pro realizaci projektu. Části projektové dokumentace mohou mít také účel prezentační, nebo orientační.

Projektová dokumentace začíná vznikat před zahájením realizace projektu a v průběhu realizace je neustále doplňována novými údaji. 

Tvorba technické dokumentace se stala v současné době nezbytnou nutností každého většího projektu. V případě složitých projektů umožňuje zachovávat pořádek při plánování a realizaci, umožňuje se k těmto projektům po čase vrátit pro účely úpravy řešení, na daném projektu následně nemusí pracovat ani sám tvůrce, ale jeho může to být jeho nástupce, kterému jeho
dokumentace poslouží jako vodítko a při práci v týmu je vhodným nástrojem pro rozdělování úkolů.
Každá projektová dokumentace by se měla skládat z několika částí:

\begitems
*Projektová specifikace
*Rozpis zdrojů (časových, materiálních, lidských, ...)
*Popis průběhu realizace projektu
*Dokumentace z testování a ladění
*Uživatelský manuál
\enditems

\sec Projektová specifikace

{\bf Projektová specifikace} je důležitou součásti projektové dokumentace. Specifikace projektu obsahuje důležité informace, které
vstupují do procesu návrhu řešení projektu. Díky projektové specifikaci uživatelé jednoduše zjistí, co lze od výsledného produktu očekávat a co nikoli.

Autorem specifikace by měl být zároveň autor resp. vedoucí projektu. Tento dokument má napomoci všem spolupracovníkům na projektu k jeho korektnímu pochopení. 

\sec Popis průběhu realizace projektu

Není možné předvídat všechny okolnosti při realizaci projektu. Z tohoto důvodu je třeba

\sec Uživatelský manuál

V případě, že produktem projektu je nějaké zařízení, nástroj, nebo jiné hmatatelné produkty, je třeba vytvořit manuál, která má za úkol případnému uživateli detajlně vysvětlit jak s tímto produktem zacházet, čeho je shopen a čeho by se měl vyvarovat. Z praktických důvodu se jedná o samostatný dokument, který je k celkové dokumentaci přikládán. 

Uživatelský manuál by mě zpravidla podrobně popisovat veškeré dostupné funkce a možnosti daného produktu a možnosti jeho využití v praxi. 

\sec Obrázková příloha

\sec Ostatní doprovodné dokumenty

\sec Shrnutí

\chap Typografie technické dokumentace

\sec Shrnutí

\chap Plánování projektu

Plánování hraje v úspěšnosti projektu důležitou roly, protože projekt je ze své definice vysoce inovativní aktivitou zahrnující činnosti, které dosud nebyly realizovány. Bez přesné konstrukce plánu bude projekt jen stěží úspěšný. Výsledkem procesu plánování by měla být projektová specifikace, která udává cíle projektu a postupy, kterými je těchto cílů dosaženo.



\sec Jméno projektu

Že každý projekt musí mít nějaké jméno je samozřejmé. Každé dobré projektové jméno ale musí splňovat určitá pravidla:

\begitems
*Název musí být jednoznační - nesmí se shodovat s jiným projektem, aby nedošlo k právnímu nebo jinému konfliktu.
*Název by měl popisovat/vystihovat funkci projektu
*Název by měl být co nejkratší. Pokud název přesahuje délku jednoho
řádku, je zřejmně něco špatně.
*Každý systém by měl mít vedle svého oficiálního jména, které splňuje body 1 - 3, měl mít i své jednoslovný název. Může to být zkratka oficiálního
názvu, nebo kódové jméno.
\enditems

Oficiální název se používá v nadpisu souvisejících dokumentů a při odkazování na projekt z jiných dokumentů. Jednoslovný název v běžné komunikaci. I když není třeba jednoslovný název definovat hned, je lepší ho ustanovit už na začátku, předejde se tak zbytečným zmatkům v budoucnu.


\sec Obsah projektové specifikace

Každá projektová specifikace musí obsahovat následující části, nejlépe v uve-
deném pořadí:

\begitems
*Název projektu
*Popis účelu projektu a jeho použití (využití) v praxi
*Zdůvodnění projektu vyjadřující jeho smysluplnnost a nutnost jeho realizace
*Popis cílové skupiny, pro kterou je produkt projektu určen
*Projektový plán - obsahuje návod jakým bude projekt realizován, slouží jako vodítko pro řízení a konstrolu projektu 
*Popis možných ryzik, která mohou realizaci projektu ohrozit a opatření, která jsou potřeba pro jejich předcházení, či nápravu přijmout.
*Předpokládaný rozpočet na realizaci
\enditems

\sec Základní otázky projektové specifikace

Při tvorbě projektové specifikace je třeba si položit několik základních otázek,
díky kterým lze získat představu o výsledné povaze projektu. Těmito otázkami
jsou:

\begitems
*{\bf CO?} - Tato otázka umožňuje zjistit co je cílem projektu, co bude tvořit, umožňuje stanovit přesné měřitelné cíle, které je třeba při realizaci splnit a dodržet.

*{\bf JAK?} - Zjištění jaké technologie a postupy je třeba použít. Celý proces je vhodné doplnit grafem ve kterém jsou zobrazeny postupné kroky výroby s použitými technologiemi a technologickými postupy. Tím je získán {\bf plán struktury projektu}. Ten upresňuje a konkretizuje myšlenky projektu.

*{\bf KDO?} - V této fázi plánování projektu je již třeba vědět, jaké jsou cíle projektu a jaké postupy jsou použity pro jejich dosažení. Z toho je možné odvodit jaké profesní znalosti a dovednosti jsou pro realizaci projektu potřebné. Projektový tým musí být sestaven tak, aby každý jeho člen měl přesně stanovenou roli, kompetence a konkrétní povinnosti, které jsou kontrolovatelné a hodnotitelné. Těm musí odpovídat stanovené kontrolní mechanismy a způsoby vykazování odvedené práce (reporting). V případě, že na projektu pracuje tým více lidí, je třeba definovat kdo bude realizovat jednotlivé části projektu. Při změnách pracovního
kolektivu v průběhu tvorby projektu se to také musí dokumentovat.

*{\bf KDY?} - datum realizace projektu je stejně klíčové jako označení začátku a konce každé na něm prováděné činnosti.

*{\bf CENA?} - tato položka by se měla skládat z rozčlenění na jednotlivé kroky projektu v kombinaci se spotřebou práce. Také by měla čítat náklady na řízení projektu a pořizování nového SW a HW.
\enditems

\sec Řízení rizika a problémů

Projekt jako takový představuje algoritmus se vstupy, který využívá nějaké vstupní prostředky díky kterým umožňuje vyprodukovat nějaký výsledek. Jako každý dobrý algoritmus musí být ošetřen vůči chybám, které mohou v jeho průběhu nastat. Špatně ošetřený algoritmus může v případě výskytu nějaké chyby skolabovat. V případě projektu by to znamenalo neuspěšně zakončený (nedokončený) projekt.

{\bf Riziko} je možné považovat za nahodilou událost jejímž následkem může být vznik nějaké škody. Naproti tomu {\bf problém} je něco špatného co již nastalo a s čím je třeba se vypořádat. V případě projektu může díky riziku či problému dojít k ohrožení jeho průběhu, deformování výsledku, ... Řízením rizik se rozumí předvídání možných rizik, která by mohla v průběhu realizace nastat a pravidelně kontrolovat jejich případný vznik. Z toho vyplývá, že je potřeba počítat se vznikem problémů, které jsou následkem nepředvídatelných rizik.

{\it Externí rizika} jsou jevy, které na projekt působí z vnějšku (politická situace, ...). Šance ovlivnit taková rizika je minimální a proto s nimi musí řešitel nebo projektový manager předem počítat a vhodně na jejich výskyt reagovat.

{\it Interní rizika} se týkají samotné náplně projektu. Mohou to být náhodné události spojené s projektovým týmem, jeho vybaveností, časovým harmonogramem, ... Řešitel nebo projektový manager může tato rizika ovlivnit a předcházet jim.

{\bf Řízení rizik} je předvídání událostí, které by mohly ohrozit úspěšné zakončení projektu a jejich eliminace. Uvědomnění možných rizik spojených s projektem napomůže ke snazší realizaci projektu, protože je snahou se těmto rizikům vyhnout, nebo podniknout takové kroky, aby tato rizika projekt neohrozila.

Při plánování jednotlivých kroků tvořící projekt je využíván plánovací nástroj pro grafické znázornění algoritmů - stavové diagramy. Stavové diagramy umožňují názorně zobrazi jednotlivé kroky vedoucí k vytvoření produktu  a zároveň vyznačit všechna možná rizika, která mohou nastat a jejich ošetření - nápravu.

\secc Proces řízení rizik

Proces řízení rizik se skládá ze tří kroků:

\begitems
*{\it Identifikace} - popis rizik, která mohou potenciálně ohrozit průběh projektu.
*{\it Plánování činností} - návrh opatření, jak rizikům předcházet nebo řešit případné problémy.
*{\it Monitorování a kontrola} - neustálý dohled v průběhu projektu tak, aby bylo možné případné odchylky korigovat a řešit potenciální problémy.
\enditems

\sec Tvorba projektového plánu

Plán projektu je součástí projektové specifikace, která udává jakým způsobem budou řešeny dané problémy při řealizaci projektu. Projektový plán vzniká postupných zpřesňováním prvotního návrhu (konceptu). Projektový plán se skládá z několika částí:

\begitems
*Plánování struktury projektu - logická a písemná struktura projetku, která slouží jako základ pro další rozhodovací procesy spojené s vývojem a realizací projektu.

*Definování struktury projektu - podle rozsahu projektru dochází k rozdělení jeho větších dílčích částí na menší, které jsou díky tomu snáze řešitelné.

*Definování činností - je nutné identifikovat činnosti spojené s realizací dílčími částmi projektu.

*Určení pořadí činností - vzhledem k tomu že by se některé činnosti mohly vzájemně podmiňovat (nelze udělat jednu činnost bez provedení druhé) je nutné přesně definovat jejich logickou návaznost.

*Odhad tvrání činností - reálný odhad trvání jednotlivých činností určuje celkový potřebaný čas na realizaci celého projektu. Přesnost jeho odhadu je závislá na zkušenostech autora.

*Tvorba harmonogramu - definované činnosti s ohledem na dobu trvání jsou zařazeny do harmonogarmu respektující definované pořadí jejich vykonávání a závislosti řešených úkolů.

*Plánování zdrojů - pro reálnost plánu je nezbytné určení, které zdroje (lidské finanční, materiálové, ...) budou v rámci projektu zapotřebí.

*Odhad nákladů - na základě seznamu plánovaných zdrojů je proveden odhad nákladů
\enditems

Aby bylo možné časový plán odhadnout s co nejmenší chybou je vhodné:

\begitems
*Rozplánovat projekt na co nejmenší časové úseky, čím kratší jsou činnosti a čím menší je jejich počet, tím může být jejich časový odhad přesnější
*Počítat s rezervou na nečekanné údálosti, komplikace, které s obtížností projektu narůstají, když je daný úkol splněn dříve je to optimistické, ale když zkončí později je to špatné
*Nepřeceňovat úroveň pracovníků, ale naopak počítat s průměrnými pracovníkami a běžnými pracovními podmínkami
*Zohlednit znalosti a zkušenosti členů projektového týmu
\enditems

\sec Shrnutí

\chap Účetnictví projektu

\sec Shrnutí

\chap Metody výzkumu a vývoje

\sec Shrnutí

\chap Řízení projektu

Úkolem řízení projektu je realizovat úkoly definované ve struktuře projektu tak, jak byly naplánovány, dosáhnout vytyčených cílů a výsledků ve stanoveném čase a se stanovenými zdroji.

Průběh celého projektu je řízen hlavním řešitelem (projektový vedoucí) nebo managerem projektu. Při řízení projetu je důležité vytvořit vnitřní organizační struktury (hierarchie projektu), stanovené termíny, ...

\sec Kontrola a monitoring projektu

Průběh celého projektu je průběžně monitorován a kontrolován. Cílem je získat přehled o aktuálním stavu projektu, porovnání současného stavu se stavem určeným plánem projektu a zjištění aktuální nebo potenciální odchylky, eliminace nepředvídatelné nežádoucí události.

Proces monitorování a kontroly lze charakterizovat jako sumu nutných aktivit směřující k ověření souladu či nesouladu současného stavu projektu s projektovým plánem. Výsledek tohoto ověřování je pak oznamován všem účastníkům s tím, že jsou sděleny případné nesrovnalosti a stanoveny příslušné kroky vedoucí k nápravě. Sledování projektu je nutné z hlediska jeho odborné náplně, časového průběhu, nákladového průběhu ale i z hlediska možných rizik. Nezbytnou součástí monitoringu a kontroly jsou pravidelné kontrolní porady vedoucích pracovníků projektu a dílčích skupin. Jedině díky účinnému průběžnému sledování postupu projektu je možné včas odhalit problémy a rizika a provést nápravné kroky.

Do kompetence hlavního řešitele projektu patří především:

\begitems
*Dohled nad projektem po jeho odborné stránce
*Sledování kapacity a výkonu projektového týmu
\enditems

Do kompetence projektového managera patří především:

\begitems
*Sledování projektu z hlediska času, kdy se dohlíží zda jsou dané problémy plněny v rámci časového plánu
*Sledování projektu z hlediska nákladů, kdy se dohlíží na vynaložené náklady v návaznosti na odhad nákladů nutných k dokončení projektu
\enditems

Výstupem kontroly a monitoringu jsou {\it monitorovací zprávy}, které jsou součástí projektové dokumentace konkrétně části dokumentující průběh realizace projektu. Tyto zprávy neslouží jen pro kontrolu stavu projektu, ale u déletrvajících projektů informují poskitovatele dotací o postupu prací a stavu čerpání zdrojů. Monitorovací zprávy mají periodický charakter, po dobu realizace projektu v pravidelných intervalech popisují průběh projektu. Formulář monitorovací zprávy obsahuje část věcnou a část finanční. Část věcná informuje o realizaci klíčových úkolů a již dosažených cílů projektu. Také informuje o změnách a jejich důvodech, které bylo nutné v projektu provést. Číst finanční slouží spolu se žádostí o platbu k vykázání uzkutečněných výdajů projektu.
\sec Práce v týmu

Pokud na projektu pracují jednotlivci, může realizace projektu trvat výrazně déle, než když na něm pracuje skupina organizovaných pracovníků - týmu. Projektový tým je skupina lidí, kteří se svými dovednostmi navzájem doplňují, přijímají účel společné práce a jsou si vědomi odpovědností za výsledek své práce. Vedení více lidí, aby pracovali na splnění daného cíle je ale složitější, protože se celá práce na projektu musí rozdělit na kroky, které mohou být vykonávány paralelni. Budování týmu má na starosti projektový vedoucí.

Pro jednotlivé členy týmu je třeba v souladu s jejich rolemi určit jejich kompetence a z nich vyplývající konkrétní úkoly, za jejichž splnění jsou zodpovědní. Členové týmu s vyššími kompetencemi a vyšší odpovědností se podílejí již na plánování projektu.

Mezi členy týmu musí probíhat neustálá komunikace, která zajistí kompaktnost projektu - jednotlivé části projektu do sebe dobře zapadnou.

Úkolem týmového vedoucího je přebírat úkoly od projektového vedoucího a projektového managera a rozdělovat je mezi členy projektového týmu. Je to jakýsi prostředník mezi vedením projektu a realizační skupinou. Dále je třeba, aby týmový vedoucí disponoval schopností motivovat jednotlivé členy svého týmu tak aby jejich produktivita byla co největší (ale ne na úkor jejich potřeb).

\sec Shrnutí

\end

