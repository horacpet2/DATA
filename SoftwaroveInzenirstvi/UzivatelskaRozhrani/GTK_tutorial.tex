\def\ddr{C:/texLib/}

\input \ddr/TeXMakro
\setAddress{\ddr}
\input \ddr/KonfiguracePaperBook

%Načtení obrázků
%\pdfximage width \the\SirkaOdstavce mm {./Obrazky/obr}


%Titulní stránka
%\VlozitDokument{TitulniStranka}

%generování obsahu
%\Obsah

\Nadpis{Úvod}

Uživatelské rozhraní je speciální typ rozhraní, které slouží ke komunikaci uživatele s programem. Jedná se o ovládací prvek většiny programů. Uživatelská rozhraní lze rozdělit do několika základních skupin:

\vskip 4mm
\bod{Uživatelské rozhraní příkazového řádku}
\bod{Textové uživatelské rozhraní}
\bod{Grafické uživatelské rozhraní}
\vskip 4mm


\Nadpis{Uživatelské rozhraní příkazového řádku}

Uživatelské rozhraní příkazového řádku je nejjednodušší typ uživatelského rozhraní, které ke své komunikaci používá textové příkazy, které jsou zadávány do příkazové řádky terminálu (terminálového emulátoru). 


\Nadpis{Textové uživatelské rozhraní}


\Nadpis{Grafické uživatelské rozhraní}

Grafické rozhraní je tvořeno grafickými prvky tzv. {\bf widgety}, které tvoří grafické okno. Grafické rozhraní je nejpřehlednějším typem uživatelského rozhraní, ale zároveň také nejpracnějším a nejnáročnějším co se výkonu týká. 

Existují dva způsoby jakým je grafické rozhraní realizováno. Jeden způsob jakým je možné vytvořit grafické rozhraní programu je, že grafické rozhraní je přímou a nedělitelnou součástí jádra programu. Tento způsob se nazývá {\bf Embedded gui} (vestavěné gui). 

Druhým způsobem je grafické uživatelské rozhraní jako nadstavba rozhraní příkazového řádku, kdy grafické rozhraní tvoří samostatnou část programu, která komunikuje s jádrem programu pomocí textových příkazů.Tento typ grafického rozhraní se nazývá {\bf Distribuovaný gui}. Tento způsob má několik výhod oproti vestavěnému gui. První výhodou je, že je možné nejprve vytvořit hlavní funkcionalitu programu a otestvoat ji v příkazovém řádku a teprve poté k ní přidat grafické rozhraní. Další možností je, že je jednoduše možné volitelně spouštet program s nebo bez grafického rozhraní. V neposlední řadě je jednoduší hlavní funkčnost programu řádně otestovat.


\Nadpis{Design zaměřený na člověka}

HUMAN CENTERED DESIGN, neboli design zaměřený na člověka je sada postupů, pro vytváření služeb a produktů zaměřených na určitou cílovou skupinu lidí. Podstatou je zapojování uživatelů do procesu návrhu a to ať potenciálních nebo zkutečných již v raném průběhu projektu. Díky tomu lze snáze pochopit problémy a požadavky uživatelů a tím lépe navrhnout výsledný produkt.

\Sekce{Fáze HCD}

HCD probíhá v několika krocích:

\vskip 4mm
\bod{Výzkum - snaha zjistit více o prostředí pro které je produkt navrhován}
\bod{Cíle a požadavky - po zjištění odpovědí na základní otázky jsou tyto informace zpracovány a díky nim je možné sestavi seznam cílů a požadavků na navrhované řešení}
\bod{Koncept - zjištění cílů a požadavků na cílový produkt je možné sestavit počáteční koncept, který demostruje základní myšlenku produktu, jedná se o sadu možných řešení ze kterých je teprve vybíráno to nejvhodnější pro realizaci}
\bod{Prototypy - na základě konceptu myšlenek jsou vybrány ty, které jsou realizovatelné a na jejich základě se vytvoří počáteční skici, makety, funkční prototypy, ... toho jak by výsledné řešení mohlo vypadat}
\bod{Testování - hotové prototypy jsou pomocí zkušební skupiny uživatelů testovány a zkoumány a na základě výsledků testování prototypů jsou vyvozeny nové závěry, které fungují jako vklad do dalšího vývoje. Cel proces se tak opakuje v cyklech, přičemž každým cyklem se návrh zpřesňuje a vylepšuje.}
\vskip 4mm

\Sekce{Výzkum}

Výzkum může mít mnoho forem o jednoduchého dotazníku po pár otázek u polední pauzy, až po hromadné konzultace cílové skupiny uživatelů. Prvním krokem výzkumu je {\bf definice otázek}, které mají za cíl zjistit neznámé údaje. Dalším krokem je {\bf výběr výzkumné metody} (dotazníky, rozhovory, ...). Následuje {\bf plán výkumu a příprava materiálu}. Čtvrtým krokem je výběr a pozvání respondentů. Následuje samotné {\bf provedení výzkumu} s cílovou skupinou uživatelů. Po provedení výzkumu je nutné {\bf analyzovat data získané z výzkumu}.

Výzkum lze rozdělit na {\bf kvantitativní} a {kvalivativní}. Tyto dvě kategorie výzkumu se liší především charakterem dat, které jsou při výzkumu získány. 

Kvantitativní výzkum je zaměřen na získávání dat (kolik, jak často, s čím, ...). Díky tomu je možné získat číselné údaje, které je možné použít do analýzy činností.

Kvalitativní výzkum je zaměřena to co, proč a jak kdo co dělá. Odpovědi odhalí motivace uživatelů potřeby a důvody různých činností.

V praxi se většinou používá kombinace obou metod výzkumu.

Při výzkumu jsou důležité určité vlastnosti psychologie lidí. Jejich postoje a chování lidí většinou protichůdné. Lidé si něco přejí, věří že je to dobré, ale stejně to ve výsledku nedělají.

\PodSekce{Metody výzkumu}

Podstatou výzkumných metod je získávání infrormací, dat o uživatelých a o prostředí daného produktu. Pro získávání informací se používá několik postupu:

{\bf Experiment} - uměle navozená situace, která by mohla v daném prostředí nastat. Následným pozorováním jsou získány informace o chování daného systému. Experiemnt by měl být opakovatelný.

{\bf Pozorování - }






\end
