\def\addr{/home/petr/MEGA/CENTRUM/texLib}

\input \addr/TeXMakro
\setAddress{\addr}
\input \addr/KonfiguracePaperBook
%\input \addr/KonfiguraceEBook
%makra

%Načtení obrázků
%\pdfximage width/height \the\SirkaOdstavce mm {./Obrazky/obr}

%Titulní stránka
%\VlozitDokument{TitulniStranka}

%generování obsahu
%\Obsah


\Nadpis{Úvod}
Inteligence

\Nadpis{Kognitivní funkce}

\Nadpis{Druhy pamětí}

\Nadpis{Vzpomínky}

Vzpomínky jsou obrazem událostí, které daný jedinec prožil. Mohou to být ale také jen myšlenky a nápady, které proběhly v mysly. Uchovávat vzpomínky je nejdůležitější vlastností možku, protože umožňuje reagovat na přítomnost a myslet do budoucnosti. Díky vzpománkám je možné si uvědomit sebe sama. 

Vzpomínky ale nejsou v mozku přesně uloženy. Jedná se pouze o rekonstrukci událostí a myšlenek. Vzpomínky jsou ovlivněny (zkresleny) různými událostmy které následovaly. Některé detajly jsou automaticky doplněny tím co je pro daného jedince nejpravděpodobnější. 


\Nadpis{Asociace}

Asociace je proces vybavování vzpomínek na základě vnějších, ale také vnitřních podnětů. Asociace má hluboký dopat na výsledný myšlenkový proces inteligence, protože umožňuje plynutí myšlenek, zrod nových nápadů, ale také vykonávání běžných aktivit. Bez asociací by nemohlo probýhat myšlení, protože, by nedocházelo k žádným podnětů, které by myšlení stymulovalo. Typickým příkladem člověka se sníženou schopností asociací vzpomínek je člověk trpící Alzheimerovou chorobou.

Alzheimerova choroba je neurodegenerativní onemocnění mozku, které se projevuje ztrátou mozkových buněk (neuronů) v některých částech mozku. Výsledkem je zhoršení intelektuálních a kognitivních shopností jedince. 



\Nadpis{Zapomínání}

Zapomínání je důležitý proces čištění paměti od nedůležitých, neaktuální a nepoužívaných informací. Díky tomu nemůže dojít v průběhu k přehlcení paměti informacemi. Zapomínání je někdy považováno za vadu paměti, která tak nedokáže dlouhodobě uchovávat informace, ale jedná se o nezbytný proces v mozku, který umožňuje zachovat přehled v uložených vzpomínkách. 

Se zapomínáním rovněž souvisí degenerativní onemocnění mozku jako je Alzheimerova choroba. Tato nemoc, ale způsobuje nepřirozeně rychlé zapomínání vlivem poškozování neurálních spojení mezi shluky neuronů tvořící vzpomínky, nebo poškození neuronů tvořící vzpomínky a tím tak způsobuje zkreslení vzpomínek a zhoršení schopnosti vytvářet nové vazby mezi neurony jehož následkem je zhoršená schopnost ukládat nové informace (vzpomínky).

Díky procesu přirozeného zapomínání je ale vědomí schopné se přizpůsobovat novým situacím a nenechat se ovlivnit předchozími zkušenostmi, které již nejsou aktuální. Zároveň umožňuje vypustit zbytečné detaily, kvůly kterým by byly vzpomínky zbytečně komplikované. 

Z toho vyplývá, že proces přirozeného zapomínání je stejně důležitý jako proces přirozeného pamatování

\Nadpis{Paměťové filtry}

Paměťové filtry jsou podvědomé algoritmy, které určují které informace si má vědomí zapamatovat a které ne. To závisí na tom, které informace paměťové filtry vyhodnotí jako důležite a zbytečné do budoucnosti a které informace v budoucnu k ničemu nebudou.

Pokud je informace vyhodnocena jako důležitá je vpuštěna přes paměťové filtry do paměti, kde je trvale (po nějakou dobu) uložena. Důležitost informace je pro každého jedince individuální(každý jedinec je jedinečný ve svých prioritách), ale nejčastěji to jsou informace spojené s nutností přežít (dědictví od předků, stresové situace které jsou spojeny s přežitím je nutné si zapamatovat pro příště), záliby (věci, které jedince baví uvolňují hormony štěstí, které způsobí dobrý pocit spojený s učením nových věcí) a věci spojené s sexualitou (touha reprodukce je obecně jeden z nejsilněších pudů jakéhokoli živého tvora a proto cokoli spojené se sexualitou vytváří silné vzpomínky).

Důležitost informace je tedy relativní, protože každý jedinec na základě svých vzpomínek přiřadí dané informace nějakou prioritu (bodové ohodnocení), které používají paměťové filtry, bez  ohledu na to zda daná informace zkutečně důležitá je nebo není (pro někoho může být důležitá a pro jiného ne).

Pokud by paměťové filtry neexistovaly paměť by byla přehlcena informacemi všeho druhu (od důležitých po triviální) a pravděpodobně by nebylo možné se v takovém množství informací orientovat.

Algoritmus paměťových filtrů je ovlivněn vzpomínkamy z minulosti, které určují na základě toho jaký druh infomace vyvolává dobré pocity, které informace se propustí do paměti a které jsou odfiltrovány.

Pokud paměťové filtry nepracují správně bývá následkem buď zhoršená shopnost pamatování a nebo chronické ukládání veškerých podnětů.

\Sekce{Proces hodnocení informací}

S paměťovými filtry nedílně souvisejí procesy, které ohodnocují příchozí informace. Na základě tohoto ohodnocení jsou pak informace propuštěny přes paměťvé filtry do paměti. Kdyby neexistovaly procesi hodnocení informací, nebylo by možné pomocí paměťových filtrů určit které informace jsou důležité a které ne.

\Nadpis{Myšlení ovlivněné pohlavím}

Pohlaví je motivace pro existenci osobnosti. Smyslem existence libovolného živého tvora je vlastní reprodukce. K tomu je ve většině připadů zaptřebí dvou jedinců a ve většině připadů se jedná o jedince opačného pohalví (hermafroditi). Od tohoto se odvijí chování.


\Nadpis{Vývoj mysli}

Mysl se v průběhu své existence neustále vyvíjí. U živých je tomu tak proto ze dvou důvodů. Prvním důvodem jsou biologické změny v mozku a druhým důvodem je sbírání zkušeností v průběhu života.

Zkušenosti jsou neúspěchy, pochybení, ohrožení na životě, ... ze kterých se daný jedinec poučil a je schopen tyto zkušenosti využít v následujcích případech.

Vývoj mysly ovlivněný okolními podněty dělá z jedince to co je $\rightarrow$ tvoří vlastní osobnost. Proto události, které jedinec zažil v raném dětství mohou výrazně ovlivnit jeho budoucí existenci (život). Některé podněty mají v raném dětství za následek tvorbu prvotních spojů mezi neurony. Pokud se dítěti nedostane dostatek podnětů může být vývoj mladé mysly pomalejší nebo se může úplně zastavit. Toto zpoždění vlivem biologického vývoje mozku již nemusí dítě dohnat.

\Nadpis{Podvědomé procesy v mozku}


\Nadpis{Rozpoznávání objektů}

\Nadpis{Sebeuvědomění}

\Nadpis{Vnímání času}

Vnímání času je vlastnost, kterou si v biologii neuvědomují všechny živé inteligentní organismy. Kromě sebeuvědomění se jedná o další schopnost vnímání. 

\end
