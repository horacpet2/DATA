\def\ddr{/home/petr/MEGA/CENTRUM/texLib}

\input \ddr/TeXMakro
\setAddress{\ddr}
\input \ddr/KonfiguracePaperBook
%\input \ddr/KonfiguraceEBook
%makra

%Načtení obrázků
%\pdfximage width/height \the\SirkaOdstavce mm {./Obrazky/obr}

%Titulní stránka
%\VlozitDokument{TitulniStranka}

%generování obsahu
\Obsah


\Nadpis{Úvod}

\Nadpis{Trasovací střely}

\Nadpis{Prototypy}

Softwarové prototypy jsou podobné jako trasovací střely s tím rozdílem, že jsou pouze dočasné a neslouží k implementaci. Softwarové prototypy se používají jako testovací prvek při zkoumání nové funkčnosti, vlastnosti, nápadu nebo technologie. Vytvoření prototypu je totiž rychlejší a jednoduší než implementace v plném rozsahu.

Díky prototypům je možné jednoduše a rychle zjistit předem neznámé chování navrhovaného systému nebo pouze jeho některých částí. Prototyp je část testovacího kódů, která neslouží k implementaci do ostré verze, ale pouze jako část dokumentace projektu prezentující jakým způsobem se chová konkrétní část. Na jeho základě se pak navrhne konkrétní část výsledného programu. 

Prototyp ale ne vždy musí být ve formě zdrojového kódu v nějakém programovacím jazyce, ale třeba ve formě nějakého druhu diagramu nebo schématické kresby grafického rozhraní. Prototypy mají za úkol odpovědět pouze na několik základních otázek a díky tomu je jejich vývoj mnohem rychlejší a tím i levnější než výsledná aplikace.

Kód prototypu může ignorovat některé nepodstatné detaily. Protože kód prototypů je dočasný nemusí být ošetřený proti zakázaným stavům, nebo dokonce může přijímat pouze několik konkrétních hodnot. Vůbec nevadí když při vložení zakázaných hodnoty prototyp spadne. Prototyp je tvořen pouze pro testovací účely, jehož cílem je zodpověděz základní otázky.  

\end
