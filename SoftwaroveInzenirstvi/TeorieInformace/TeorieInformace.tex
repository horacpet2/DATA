%preambule
\input D:/MEGA/CENTRUM/texLib/TeXMakro
\input D:/MEGA/CENTRUM/texLib/KonfiguracePaperBook
%\input C:/Users/HORACEK/Documents/MEGA/CENTRUM/texLib/KonfiguracePaperBook/KonfiguraceEBook

%\input /home/petr/bin/texLib/TeXMakro
%\input /home/petr/bin/texLib/KonfiguracePaperBook

%Načtení obrázků
%\pdfximage width/height \the\SirkaOdstavce mm {./Obrazky/obr}


%Titulní stránka
%\VlozitDokument{TitulniStranka}

%generování obsahu
\Obsah

\Nadpis{Úvod}

Teorie informace je vědní obor zabívající se popisem informací, jejich uchováváním přenášením a reprezentací. K tomuto účelu využívá matematické nástroje a teorie. 

V průběhu dějin byly vytvářeny nejrůznější informace - obrázkové, textové, zvukové, ... Základem teorie informace je způsob jakým různé typy informací popsat pomocí jednotného způsobu. K tomu slouží základní informační jednotka - bit (zkratka anglických slov binary digit). Díky poznatkům teorie informace jsou informace rozbity na elementární datové částice - bity, kterými je možné danou informaci pospat.  Díky datové jednotce bit je možné popsat dříve rozdílné typy dat. Jednotný způsob popisu informací umožňuje také jednotný způsob přístupu, záleží pouze o způsobu jejich reprezentace.

Bit vznikl postupným redukováním informací na dílčí informace až na elementární dále již nedělitelná data. Bit umožňuje nejjednoduší způsob popisu informací. Bit je založen na jednoduché hádací hře. Jeden člověk si myslí nějakou věc (informaci) a druhý hádá co je to za věc, tím že se ho ptá na charakteristické rysi dané věci. Ten však může na otázku odpovědět pouze ano nebo ne. To koresponduje s vlastnostmi elementární datové jednotky bit, která také může nabývat pouze dvou určitých stavů, které jsou označovány jako 1 (pravda) nebo ne (nepravda). 

V hádací hře je ale potřeba se umět správně ptát, aby odpověď na otázku ano nebo ne měla určitý význam. Proto je nutné předem definovat množinu základních otázek, kterými je možné daný druh informace popsat. Díky tomu je možné pomocí odpovědí ano nebo ne na základní otázky popsat libovolnou informaci téhož druhu. Pro jiný druh informace je nutné definovat jinou množinu základních otázek, ale způsob reprezentací odpovědí zůstává pořád stejný. V praxi je hádač, který pracuje s informacemi člověk, který hádá o jaké informace se jedná a paměť je člověk, který odpovídá na položené otázky. Na základě toho jak paměť odpoví jsou odpovědi interpretovány v obraz na monitoru, zvuk, pohyb motoru, instrukce pro součet dvou čísel, ...

Díky teorii informace došlo k velkému rozvoji informatiky, která se zabývá způsobu reprezentace, uchovávání a přenosu nejrůznějších druhů informací. Informace je výždy svázána s nějakým nosičem, který informaci uchvává. Informace je vždy svázána s nějakým fyzickým nosičem a nelze uchovávat informaci bez fyzické reprezentace. 

Do teorie informace patří ne jen způsob reprezentace dat, ale také práce s nimi - algoritmus. Algoritmu je také určitá forma informace, kterou je možné zaznamenat pomocí dat.


\end
