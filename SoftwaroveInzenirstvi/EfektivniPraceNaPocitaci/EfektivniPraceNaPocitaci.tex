%\def\addr{/home/petr/.texLib}
\def\addr{C:/texLib}

\input \addr/TeXMakro
\setAddress{\addr}
\input \addr/KonfiguracePaperBook
%\input \addr/KonfiguraceEBook
%makra

%Načtení obrázků
%\pdfximage width/height \the\SirkaOdstavce mm {./Obrazky/obr}

%Titulní stránka
%\VlozitDokument{TitulniStranka}

%generování obsahu
\Obsah


\Nadpis{Úvod}



\Nadpis{Pracovní prstředí}

Pracovní prostředí je místnost, ve určená k práci spolu s jejím vybavením. Tato místnost by měla být přiměřeně veliká, dostatečně osvětlená, dobře větraná a odhlučněná. Dodržení těchto podmínek je jednou z prevencí předcházení zdravotních obtíží spojených s dlouhodobou sedavou činností. Při jednostranné zátěži, kterou působí sedavá činnost jsou kladeny zvýšené nároky na některé části těla, které jsou tímto přetěžovány a naopak omezována činnost jiných. Vzniká tak svalová disbalance, která se může projevit bolestí.

Rozvržení pracovního prostoru (kanceláře, stolu, ...)  má veliký dopad na kvalitu a rychlost odvedené práce. Pro dosažení maximýlního výkonu při vykonávání činností spojených s počítačem je tedy nutné zvolit správnou strategii ne jen při samotné práci, ale také při plánování činností. Plánování a samotnou činnst ovlivňují vnější vlivy, které člověka u počítače nesmí negativně ovlivňovat, ale mají mu naopak dopomáhat k lepším výsledkům.

\Sekce{Kancelářská židle}
Dobrá kancelářská židle se posuzuje podle několika zásad:
\vskip 4mm
\bod{Dostečná stabilita - židle musí být na nejméně 5-ti bodové základně}
\bod{Odpružení sedadla - při sedání musí tlumič dobře tlumit tvrdší dopad těla}
\bod{Výška sedací plochy musí být nastavitelná}
\bod{Plocha sedadla musí být prostorná a pohodlná}
\bod{Přední hrana sedadla by měla být zaoblená}
\bod{Opěradlo židle by mělo být nastavitelné ne jenom výškově ale také v předozadním směru}
\vskip 4mm

\Sekce{Pracovní stůl}

Výška pracovního stolu by měla být taková, aby při sezení na židly a položení prstů na klávesnici svýraly lokty úhel $90ˆ\circ$ a zárovň nesmí docházet ke zvedání ramen a hrbení zad. Proto dobrý stůl by měl mít možnost výškového nastavení pracovní desky.

\Sekce{Rozvržení pracovního stolu}

Pracovní stůl je jednou z nejdůležitějších částí pracovního prostoru programátora. Většina práce se totiž odehrává právě na jeho pracovním stole a to ať už na jednom, nebo v případě více pracovišť na více pracovních stolech. Rozvržení pracovního stolu silně závisí na jeho rozměrech, umístění v místnosti a na samotném rozvržení dané místnosti. V případě, že je pracovní stůl určen pro práci na počítači, neměl by být určen pro jiné činnost - neměl by být kombinován s jinými typy pracovních stolů. To totiž nutně znamená reorganizaci pracovního stolu (dočasné přemístění, přeskupení předmětů na stole, přinesení jiných předmětů) a z tohoto důvodu je jen velice obtížné udržet na takovém stole nějaký řád.

V případě, že je pracovní stůl zaplněn zbytečnostmi, papíry, zbytky po občerstvení, ... je těžké a zdlouhavé na něm cokoli najít, nebo odložit. Prvním pravidlem tedy je, aby na pracovním stole bylo co nejméně věcí. Mezi nezbytné věci tedy patří:
\bod{{\bf Počítač} - v případě stolního počítače je to hlavně monitor, často se již pracuje i v kanceláři na přenosném laptopu }
\bod{{\bf Klávesnice} - klávesnice se může nacházet ne jen u stolního počítače, ale také u přenosných laptopů, kde je použita jako externí polohovatelná klávesnice pro usnadnění psaní u nevhodně umístěných laptopů}
\bod{{\bf Počítačová myš} - počítačová myš je již nezbytnou součástí většiny počítačů (kromě tabletů) a je nutné s nimi počítat na pracovním stole}
\bod {{\bf Osvětlení} - v případě, že nedostačuje osvětlení denním světlem (které je nejvhodnější, protože neunavuje tolik oči), nutné vypomoci stolní lampičkou}
\bod{{\bf Psací potřeby} - při práci programátora je velice často potřeba psát různé poznámky, nebo grafy, které umožní uspořádat myšlenky, nebo dočasně zaznamenat různé nápady, nebo povinnosti}

\Sekce{Důvod nepořádku a chaosu}

Nepořádek a chaos na pracovním stole a obecně v místnosti má pouze jednu jedinou příčinu. Věci na nemají přesně stanovené umístění. Pokud věci mají přesně stanovené umístění, člověk se je snaží na tato místa odkládat a proto na tato místa neodkládá žádné jiné věci. Tento princip má pouze jeden problém. V případě, že programátor (obecně kdokoli) pracuje s větším množstvím věcí, které nemají přesně stanovené umístění, má tendenci tyto věci odkládat i na místa, která již jsou určena jiným věcem. Jednotlivé předměty se tedy začnou postupně hromadit a tím vzniká opět původní nepořádek a chaos na pracovní ploše.


\Sekce{Pomůcky pro uspořádání pracovního prostoru}
V případě, že je nutné uložit více předmětů než na kolik je možné rozvrhnout pracovní prostor, je možné si částečně vypomoci některými zlepšeními a optimalizacemi. 

Prvním pomůckou jsou {\bf šuplíky} nebo {\bf regály}, do kterých je možné skládat a organizovat písemné dokumenty a jiné předměty. 


\Nadpis{Hygiena u počítače}

Hygiena práce u počítače umožňuje šetřit fyzické i psychické síly a tím i lidské zdraví. Tím umožňuje zvýšit pracovní výkonost a produktivitu práce a zároveň tak může i snížit čas, který je potřeba strávit u počítače.


\Sekce{Font}

Výběr vhodného fontu pro zobrazování textu na počítačovém monitoru výrazně ovliňuje ne jen rychlost jakou je člověk schopen číst daný text, ale také je tím ovlivněno namáhání zraku a psychických sil. Pokud je zvolen nevhodný font, kdy člověk musí vynaložit více úsilí pro rozeznávání jednotlivých znaků dochází ke zpomalení rychlosti čtení textu. 

Mozek je více namáhán ne jen rozeznáváním znaků, ale také jednotlivých slov. Při běžném čtení mozek přečte pouze několik prvních a několik posledních písmen slova a podle kontextu odvodí o jaké slovo se jedná. Pokud ale má ale mozek více práce s rozlišováním slov tento proces se výrazně zpomalí.

Kromě zvoleného fontu ovlivňuje čitelnost textu také jeho velikost. Text na monitoru by neměl být ani příliš malý ani příliš velký. Pokud font bude příliš velký dojde vlivem rozlišení monitoru a lidského oka ke splívání jednotlivých znaků a tím i k jejich možné záměně. Tyto zásady hraničí s typografickými pravidly, které se zabývají psychologií fontů. Někeré fonty  mají estetický účel a jiné fonty mají být pouze dobře čitelné.

Znaky špatně čitelného fontu jsou tenké linie, ozdobné fonty (slouží pouze pro nadpisy ne pro dlouhé texty), ...

Pro psaní dlouhých textů jako jsou zdrojové texty, konfigurace, poznámky, ... slouží speciální fonty, které jsou navržené pro snadné čtení.


\Sekce{Barevný kontrast}

Rychlost čtení (obecně orientace na obrazovce) neovlivňuje jen zvolený font, ale také zvolený barevný kontrast pozadí a popředí.

Při střídavém pohledu z tmavší části pracovní plochy na světlejší působí zátěž na nitrooční svaly a podílí se velkou mírou na pocitu unavených očí. Proto je rozumné osvětlit celou místnost rovnoměrným osvětlením, které není příliš odlišné od svítivosti monitoru. Takové osvětlení nevrhá žádné odlesky od monitoru. Ideální osvětlení je takové, které difuzně osvětlí celou místnost stejně intenzivně jako světlo monitoru.

\Sekce{Počítačový monitor}

Počítačový monitor by měl být v určité vzdálenosti od oka v závislosti na jeho úhlopříčce. Pokud je počítačový monitor příliš blízko oku, bude docházet k příliš častým pohybům oka po ploše monitoru a tím bude zatěžován okohybný systém oka. 

U počítačového monitoru hraje roly také obnovovací frekcence obrazu. Platí čím vyšší obnovovací frekvence obrazu monitoru, tím čistší obraz a tím méně je oko namáháno

\Sekce{Zdravotní potíže vznikající při sedavé činnosti}

Při dlouhodobé sedavé činnosti trvající několik týdnů, měsíců popřípadě nepřetržitě mohou bez vhodné prevence doprovázet různé zdravotní komplikace. Tyto komplikace se neprojevují pouze fyzicky ale také psychicky. Vhodnou prevencí se lze těmto obtížím vyhnout a udržet si pevné zdraví nezávisle na dobu strávenou u počítače. 

Fyzické zdravotní obtíže lze rozdělit na několik čístí:

\vskip 4mm
\bod{Dolní končetiny}
\bod{Klouby dolních končetin}
\bod{Pánev}
\bod{Hrudník}
\bod{Krk a hlava}
\bod{Zápěstí}
\vskip 4mm

\Sekce{Dolní končetiny}

Dolní končetiny jsou vlivem dlouhodobého sezení nebo naopak dlouhodobému stání na místě vystaveny riziku vzniku křečových žil tlaku v lýtkách a chodidel a otoky kolem kotníků. Hlavní příčinou je snížení žilní návrat krve k srdci a krev se žilách městná (žilní městky neboly křečové žíly). Dalším faktorem je plochonoží, které je způsobeno nevhodným obutím. Tím je způsobeno nedostatek podnětů pro svaly chodidla. Následkem je snížení klenby nohou a tím i změnou anatomických poměrů. Tím začnou být nadměrně namáhány některé svaly v těle a naopak jiné jsou zanedbávány. To může být příčinou bolesti i jiných částí těla.

Řešením je koupě vhodné obuvy (nesmá mačka na špičce dřít na okrajích, boty musejí být prodyšné aby se nohy nepařily vnik plísní, s mírným podpatkem 2-4 cm, naprosto nevhodná je absolutně rovná podrážka nebo příliš vysoký podpatek), speciálních anatomických vložek, masážní podložka na podlahu nebo chůze na boso venku (nejlépe po oblázkách).

\Sekce{Klouby dolních končetin}

Klouby nohou, obzvláště kolena a kyčle reagují na dlouhodobý sed bolestivým zatuhnutím, které je umocněno zkracujícími se svaly. Tendeci ke zkracování mají hlavně svaly na zadní straně lýtka, svaly na vnitřní a zadní stěně stehen. Řešení a prevence je nesedět celý den a občasně protáhnout nejen zkracující se svaly na dolních končetinách. Výbornou kompenzací a prevence zkracování svalů a bolesti kloubů je rekreační běh (jogging).

\Sekce{Pánev}

V oblasti pánve je typickou obtíží bolest v kříži, způsobená přetížením z dlouhodobého nevhodného sedu a nepoměreme mezi břišními svaly, které bývají slabé a svaly zad, které bývají zkrácené. Kompenzací a prevencí je opět nesedět celý den v kuse a dodržovat vhodný sed. 

\Sekce{Hrudník}

V oblasti hrudníku se mohou vyskytovat bolest mezi lopatkami nebo bolesti v oblasti spojení žeber s hrudní kostí. Příčinou bývá nesprávný sed (shrbená záda), kdy dochází k přetěžování kloubních spojení hrudního koše. Mezilopatkové svalstvo ochabuje, prsní se zkracují a to ještě více shrbenou polohu podporuje. Dalším důvodem může být nevhodně umístěný monitor, díky kterému je člověk nucen se nepřirozeně natáčet celým hrudníkem. 

Kompenzací a prevencí je dechové cvičení hlubokoho nádechu a výdechu. Tím jsou procvičena vštšina hrudních svalů. Dalším kompenzací a prevencí je opět pravidelný běh.

\Sekce{Krk a hlava}

Bolesti šíje a hlavy jsou vedle bolesti křívé oblasti, snad typem obtíží dlouhodobé sedavé činnosti. Velmi často jsou způsobena přetížením šíjového svalstva. Bolest hlavy může být způsobena i únavou očí z celodenního sledování obrazovky, celkovou únavou a nesprávnou polohou hlavy.

Na bolesti šíje se podílí nevhodně umístěný monitor a klávesnice, nevhodná židle i výška stolu. Řešením tak je úprava pracovního prostředí a procvičování očí a procvičování šíjových svalů (běh). Bolest hlavy může být způsobena i jinými faktory a při přetrvajících potížích je nutné vyhledat odbornou pomoc.

\Sekce{Zápěstí}

Problémy se zápěstím vznikají nevhodné strnulé pozice rukou a předloktí při psaní na klávesnici či jiných činnostech při sedavé činnosti. Dlouhodobé přetěžování svalů a šlach v oblasti zápěstí vede ke vzniku otoků a bolestivosti. Protože v oblasti zápěstí se nacházejí v relativně malém prostoru svaly, šlachy, cévy a nervy je velká pravděpodobnost, že právě zde projeví zdravotní potíže. V důsledku otoku se sníží průtok krve a dojde k útlaku nervů což se odborně nazývá syndrom karpálního tunelu. 

Aby se předešlo těmto zdravotním komplikacím je nutné dbát na pracovní polohu a pravidelně ulevovat zápěstí. K tomu slouží různé pomůcky jako powerball nebo posilovací pružiny, které procvičí a uvolní svaly v zápěsti.

\Sekce{Zrak}

Při dlouhodobé práci u počítače jsou oči jednostraně zatěžovány a dochází k jejich únavě, která se může projevovat různými způsoby.

{\bf Zraková ostrost} je rozlišovací schopnost zrakového aparátu. Při přeostření na blízko je nutné cca 3 dioptrie přidat buď pomocí brýlí nebo oční akomodací

{\bf Akomodace} je schopnost aktivně vidět ostře ne jen do dálky ale také do blízka. Ta je sprostředkováno nitrooční svalovou činností. Tato schopnost klesá po 40. roku života.

{\bf Motilita} je schopnost správně hýbat očima. Podstatná je dokonalá souhra obou očí díky 12 okohybným svalům.

{\bf Fotoreakce} je aktivní schopnost oční zornice zúžit se při silnějším osvětlení a rozšířit se ve tmě.

{\bf Brýlová vada} je vada při které je možnosti otře vidět do dálky umožněna pouze za pomocí korekce brýlemi.

{\bf Presbiopie} je ztráta akomodace ve stáří.

{\bf Prostorové vidění} je shopnost očí a mozku vnímat okolní svět prostorově. Nezbitnou součástí je dokonalé ostré vidění obou očí současně.

\PodSekce{Odlišnost zrakové zátěže}

Při {\bf čtení papírových knih} je vzdálenost od oka stále více méně stejná a tak oční akomodace není příliš využívaná. Pohyb očí je převážně horizontální a prostorové vidění není příliš namáháno. Osvětlení cílové části textu se neměnní a odlesky od papíru nejsou většinou žádné. Závěrem je, že čtení není pro oči nijak zvlášť zatěžující.

Při {\bf sledování televize} dochází k mírným odlišnostem oproti čtení knih. Vzdálenost oka od obrazovky je větší a proto oko téměř nezapojuje akomodaci. V případě sledování televize v přítmí je rozdíl mezi září obrazovky a okolím velmi výrazný. Ani zde není námaha zraku příliš velká.

Při {\bf práci na počítači} jsou nejméně 3 místa, které člověk upočítače průběžně sleduje. Tato místa jsou monitor počítače, klávesnice (myš) a textové materiály, ze kterých jsou opisovány informace do počítače. 

Tyto předměty jsou vštšinou v různé vzdálenosti od oka, proto oční akomodace musí usilovně pracovat. Zároveň není různá jen relativní vzdálenost těchto předmětů, ale také jejich sklon a nutnost natáčet hlavu a oči. Zde je velmy namáhána krční páteř a okohybný systém. Zároveň musejí při těchto pohybech spolupracovat různá centra mozku, která jsou tím nadměrně namáhána a tím dochází k únavě.

\Podsekce{Uvolňovací cviky}

Pro uvolnění oka se v pohodlném sedu zakryjí obě oči dlaněmi (pravá ruka pravé oko, levá ruka levé oko), přičemž se dlaně samotného oka nedotýkají. Čím větší tmu oko vidí tím lépe. Při klidném dýcházní se oči příjemně uvolní. Takto přikryté oči se nechají několik minut. Toto cvičení by se mělo provádět několikrát deně.

Pro uvolnění stažených očních svalů se v pohodlném sedu drží hlava vklidu a nejprve se očima pomalu pohybuje 6x nahoru a dolů. Poté stejným způsobem doprava a doleva. Následuje kroužení očí v jednom směru a následně v opačném směru.


Unavené oči se příjemně osvěží opláchnutím chladnou vodou. Pozor na chlór v kohoutkové vodě, aby voda neměla pro oči opačný účinek. Takto se oči proplachují několikrát po sobě, aby se hydratovaly a prokrvily.

Pro procvičování a udržení oční akomodace slouží cvičení při kterém se asi  20 cm před očima drží ukazováček pravé ruky ve vzpřímené pozici. Následně se 10x po sobě střídavě přeostřuje mezi ukazováčkem ruky a vzdálným předmětem před očima. Je důležité tento cvik provádět pečlivě s dokonalým přeostřením. Tento cvik se provádí tak často jak jen je to možné. 

\Nadpis {Organizace práce}

Oragnizace práce slouží k zefektivnění a urychlení vykonávané práce. 
   

\end
