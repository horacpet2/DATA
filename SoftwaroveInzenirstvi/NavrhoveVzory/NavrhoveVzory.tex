\def\ddr{/home/petr/MEGA/CENTRUM/texLib}

\input \ddr/TeXMakro
\setAddress{\ddr}
\input \ddr/KonfiguracePaperBook
%\input \ddr/KonfiguraceEBook
%makra

%Načtení obrázků
%\pdfximage width/height \the\SirkaOdstavce mm {./Obrazky/obr}

%Titulní stránka
%\VlozitDokument{TitulniStranka}

%generování obsahu
\Obsah


\Nadpis{Úvod}


\Nadpis{Návrhový princip DRY}
Návrhový princip DRY - Don't Repeat Yourself sděluje jednoduchou myšlenku. "Neopakovat se". To znamená nevytvářet duplicitní informace v softwarovém projektu. Jedná se sice o pravidlo z kategorie "To je přece jasné", ale jeho správným použitím lze získat kvalitnější kód a ušetřit tak spoustu času při programování a testování. 

Návrhový vzor DRY doslova říká:

{\bf Každá dílčí znalost musí mít v systému pouze jedinou, jednoznačnou, směrodatnou reprezentaci}



\end
