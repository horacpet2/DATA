\def\ddr{/home/petr/.texLib}

\input \ddr/TeXMakro
\setAddress{\ddr}
%\input \ddr/KonfiguracePaperBook
\input \ddr/KonfiguraceEBook
%makra

%Načtení obrázků
%\pdfximage width/height 	he\SirkaOdstavce mm {./Obrazky/obr}

%Titulní stránka
%\VlozitDokument{TitulniStranka}

%generování obsahu
\Obsah


\Nadpis{Úvod}

Elektrotechnika je vědní obor, který se zabývá spotřebou elektrické energie a zařízeními, které se pro tento účel používají. Elektrotechnika se dělí na silnoproudou a slaboproudou. {\bf Silnoproudá elektrotechnika} se zabývá výrobou a rozvodem elektrické energie v elektrárnách. {\bf Slaboproudá elektrotechnika} se zabývá komunikačními, výpočetními a řídícími zařízeními, které pracují na nízkém napětí a při nízkém proudu.

\Nadpis {Elektronová teorie}

\Sekce{Elektrický náboj}

Látky obsahují elektricky nabité částice - protony a elektrony. Tyto částice nesou elektrický náboj, který se označuje v jednotkách SI písmenem {\it Q}. Jednotku elektrického náboje je {\it Coulumb} {\it kůlumb} označovaný písmenem {\it C}.

Částice díky svým nábojům na sebe navzájem působí elektrickými silami. Náboj pronotnu a elektronu je nejmenší náboje, které mohou existovat. Nazývají se {\bf elementární náboje} a jsou rovny hodnotě:

$$ e = 1,6 \cdot 10^{-19}$$

Jeho hodnota se nemění vnějšími vlivy a nelze jej oddělit od částic, které jej nesou. Náboj protonu a elektronu je sice stejný, ale vzájemně inverzní. Proto je-li počet protonů a elektronů v daném tělese stejný a jsou-li rovnoměrně rozloženy, nejsou pozorovány žádné elektrické síly kolem tělesa. O takovém tělese se říká, že je {\bf elektricky neutrální}.


\Sekce{Elektrické napětí}

\Sekce{Elektrický proud}

\Sekce{Elektrická vodivost}



\Nadpis{Rozdělení látek podle elektrické vodivosti}

Látky mají různou vnitřní strukturu a chemické složení a tím i různou schopnost vést elektrický proud. Podle schopnosti vést elektrický proud se látky rozdělují do několika skupin:

\vskip 4mm
\bod{Vodiče}
\bod{Polovodiče}
\bod{Izolanty}
\vskip 4mm

%todo podrobnější popis vodiče a izolantu (nevodič)

\Sekce{Vodiče}

Materiály, které dobře vedou elektrický proud se nazývají {\bf vodiče}. V elektrotechnice se nejčastěji používají měď a hliník. 

V kapalinách nejsou nosiči náboje elektrony, ale ionty (nabitý atom nebo molekula). Kapalné vodiče se označují společným názvem {\bf elektrolyty}. 

Atomy pevných vodivých materiálů mají ve valenční vrstvě elektron, který se může uvolnit do valenční vrstvy sousedního atomu. Tyto elektrony jsou označovány jako {\bf volné elektrony}. 

\Sekce{Izolanty}
Pevné látky, které nevedou elektrický proud se označují jako {\bf izolanty}. Elektrony ve valenční vrstvě atomů izolantů jsou silně vázány a nemohou tak přeskočit do vrstvy valenční sousedního atomu. Tyto elektrony se nazývají {\bf vázané elektrony}. Izolanty tak neobsahují žádné částice, které by sprostředkovaly průchod proudu. Mezy izolanty patří sklo, keramika nebo plasty. Dokonalé izolanty ale neexistují. Látky většinou obsahují vady a nečistoty, které (příměsy), které mohou vést k tzv. {\bf průrazu izolantu}. To znamená, že se v izolantu vytvoří vodivý kanál, kterým může proud procházet.


\Sekce{Hustota elektrického proudu}

\Nadpis{Elektrický obvod}
Elektrický obvod je uzavřená vodivá smyčka, která se skládá ze zdroje napětí, spotřebiče, elektrického vedení a případně dalších součástek.

{\bf Spotřebič} je každé elektrické zařízení ve kterém se účelně mění elektrická energie v jinou požadovanou energii. Každý spotřebič je kontruován na určité napětí. Toto napětí se nazývá {\bf jmenovité napětí} a označuje se písmenem $U_n$. Při překročení jmenovitého napětí se spotřebič poškodí a při nižším napětí zase nepracuje správně (a může se také poškodit). 

Vedení elektronů a tedy elektrického proudu mezi zdrojem napětí a spotřebičem obstarávají vodiče. Vodiče mají většinou kruhový tvar, ale mohou mít také tvar obdélníku a speciální případem je pak plošný spoj na elektronické desce.

Elektrický obvod je {\bf uzavřený} právě tehdy když všechny součásti obvodu jsou spojeny tak aby jimi mohl protékat proud. V obvodu stejnosměrného proudu vystupují elektrony ze záporné svodky zdroje napětí do vodiče, procházejí vodiček do spotřebiče a vystupují ze spotřebiče opět do vodiče, který je vede na kladnou svorku zdroje napětí. kl, 


\Sekce{Ohmův zákon}

\Nadpis{Kirchhofovy zákony}

\Nadpis{Zdroj napětí a proudu}


\Nadpis{Rezistor}


\Nadpis{Kondenzátor}

\Nadpis{Cívka}


\Nadpis{Analýza lineárních elektrických obvodů} 




\Nadpis{Polovodiče}

\Nadpis{Dioda}

\Nadpis{Tranzistor}


\Nadpis{Střídavý proud}


\Nadpis{Indukce}

\Nadpis{Transformátor}

\end


