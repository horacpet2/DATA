\def\ddr{/home/petr/MEGA/CENTRUM/texLib}

\input \ddr/TeXMakro
\setAddress{\ddr}
\input \ddr/KonfiguracePaperBook
%\input \ddr/KonfiguraceEBook
%makra

%Načtení obrázků
%\pdfximage width/height 	he\SirkaOdstavce mm {./Obrazky/obr}

%Titulní stránka
%\VlozitDokument{TitulniStranka}

%generování obsahu
\Obsah


\Nadpis{Úvod}

Elektrotechnika je vědní obor, který se zabývá spotřebou elektrické energie a zařízeními, které se pro tento účel používají. Elektrotechnika se dělí na silnoproudou a slaboproudou. {\bf Silnoproudá elektrotechnika} se zabývá výrobou a rozvodem elektrické energie v elektrárnách. {\bf Slaboproudá elektrotechnika} se zabývá komunikačními, výpočetními a řídícími zařízeními, které pracují na nízkém napětí a při nízkém proudu.

\Nadpis{Rozdělení látek podle elektrické vodivosti}

Látky mají různou vnitřní strukturu a chemické složení a tím i různou schopnost vést elektrický proud. Podle schopnosti vést elektrický proud se látky rozdělují do několika skupin:

\vskip 4mm
\bod{Vodiče}
\bod{Polovodiče}
\bod{Izolanty}
\vskip 4mm

\Sekce{Vodiče}

Vodiče jsou materiály, které dobře vedou elektrický proud.

\end
