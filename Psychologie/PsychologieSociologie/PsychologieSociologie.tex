\def\ddr{/home/petr/MEGA/CENTRUM/texLib}

\input \ddr/TeXMakro
\setAddress{\ddr}
\input \ddr/KonfiguracePaperBook
%\input \ddr/KonfiguraceEBook
%makra

%Načtení obrázků
%\pdfximage width/height \the\SirkaOdstavce mm {./Obrazky/obr}

%Titulní stránka
%\VlozitDokument{TitulniStranka}

%generování obsahu
\Obsah


\Nadpis{Úvod}

Psychologické vědy mají problém s kvantifikací dat.

Sociologie je věda o společnosti a interakce jedinců.
Sociální jev je vše co vzniká ze vzájemného působení lidí.
Sociální struktura 
Psychologie se zaměřuje na psychikou jedince, která je výsledkem mozkové činnosti

Sociální psycholigie je kombinací psycholigie a sociologie

můltiparadigmatická věda je věda, která podporuje více pohledů na problematiku.


ID - zcela nevědomá iracionální složka vědomí

EGO - vědomá část

behaviorismus 




\end
