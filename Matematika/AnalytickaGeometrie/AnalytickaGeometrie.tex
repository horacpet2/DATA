
%\def\addr{D:/MEGA/CENTRUM/texLib}
\def\addr{/home/petr/.texLib}

\input \addr/TeXMakro
\setAddress{\addr}
%\input \addr/KonfiguracePaperBook
\input \addr/KonfiguraceEBook
%makra


%Načtení obrázků
%\pdfximage width/height \the\SirkaOdstavce mm {./Obrazky/obr}
\pdfximage width \the\SirkaOdstavce mm {./Obrazky/CiselnaOsa.png}
\pdfximage width \the\SirkaOdstavce mm {./Obrazky/PolarniSoustavaSouradnic.png}
\pdfximage width \the\SirkaOdstavce mm {./Obrazky/kvadrantyKartezskySouradnySystem.png}
\pdfximage width \the\SirkaOdstavce mm {./Obrazky/KartezskysouradnySystemProstor.png}
\pdfximage width \the\SirkaOdstavce mm {./Obrazky/KartezPolar.png}

%Titulní stránka
%\VlozitDokument{TitulniStranka}

%generování obsahu
\Obsah


\Nadpis{Úvod}

Analytická geometrie je část geometrie, která se zabývá popisem geometrických útvarů pomocí algebraických metod {\bf vektorové algebry} a {\bf metody souřadnic}. Těmto metodám se souhrnně říká analytické a z toho následně plyne název {\bf analytická geometrie}. Podstatou této matematické disciplíny je převedení geometrické úlohy na algebraickou, často na {\it soustavu rovnic}.

{\bf Analytickým vyjádřením geometrického útvaru} se nazývá {\bf vztah} (relace), který splňují právě jen souřadnice bodů tohoto útvaru a žádný jiný. Výrok „útvar definovaný množinou bodů {\it U} má analytické vyjádření {\it V} (předpis, ze kterého je možné získat množinu bodů daného geometrického útvaru U)“ tedy znamená:

{\it Bod $X \in U \Rightarrow$ souřadnice bodu {\bf X} splňují vztah {\bf V} (bod X je bodem daného geometrického objektu právě když splňuje vlastnost V), platí tedy konjunkce mezi těmito výroky:}

\vskip 4mm
\bod {Souřadnice každého bodu {\it X} útvaru {\it U} splňují analytické vyjádření {\it V}.}

\bod {Každý bod {\it X} splňující analytické vyjádření {\it V} je bodem útvaru {\it U}.}

\vskip 4mm

Analytické vyjádření útvaru {\it U} má nejčastěji podobu rovnice nebo nerovnice, respektive jejich soustav s případnými doplňkovými podmínkami. Proměnnými v analytickém vyjádření útvaru jsou hodnoty na souřadnicových osách souřadnicového systému ve kterém se daný útvar {\it U} nachází (x, y, z, …).

Analytická geometrie má velký význam pro celou matematiku, ale ne jen pro ni. Analytická geometrie umožňuje rozvíjet různé matematické metody, jako například matematická analýza, ale má také velký význam v počítačové grafice a fyzice především v mechanice. Obecně lze říci, že analytická geometrie je důležitá pro všechny obory, které se určitým způsobem zabývají prostorovými objekty, nebo je nepřímo používají.

\Nadpis {Souřadnicové soustavy a body}

{\bf Soustava souřadnic} (též {\bf souřadnicová soustava} či {\bf systém souřadnic}) umožňuje jednoznačně popsat polohu bodu pomocí čísel jakožto souřadnic čili {\bf koordinát}. Geometrické úlohy je pak možno řešit matematickými (analytickými) prostředky, což je základ analytické geometrie. Polohu bodu na přímce určuje jedno (reálné) číslo, v rovině dvě, v prostoru tři čísla atd. Obecně je k určení polohy bodu v n-rozměrném prostoru třeba {\it n} čísel, která tvoří uspořádané n-tice (čti entice), neboť na jejich pořadí záleží.

Druh souřadné soustavy, tedy její uspořádání a způsob popisu pohlohy také určuje způsob matematického popisu gemetrických objektů - matematický tvar vzorců.

Protože každý geometrický útvar je reprezentován nějakou množinou bodů, které mají vůči sobě určité vzdálenosti, je třeba definovat nějaký jeden bod, který bude mít vždy stejnou polohu a díky kterému je možné vztahovat polohu všech bodů. Tento bod se nazývá {\bf počátek}, nebo {\bf střed} souřadnicové soustavy. Každý bod reprezentující daný geometrický útvar má určitou vzdálenost a směr od tohoto počátečního bodu. Způsob jakým tuto vzdálenost a směr vyjádřit je pomocí souřadnic bodu a tomu je třeba zavést souřadnicovou soustavu. Souřadnicová soustava je dohodnutý systém „adresování“ jednotlivých bodů v daném prostředí (dané prostředí může mít různé množství dimenzí - rozměrů).

Soustavu souřadnic tvoří:

\vskip 4mm
\bod{příslušný počet {\bf souřadnicových os}, které se zpravidla označují malými písmeny v abecedním pořadí: x, y, z, také lze např. i, j, k, l, m, n, nebo a, b, c, d, ...}
\bod{{\bf počátek}, kde se všechny osy protínají, mají společný bod a zároveň hodnoty všech souřadnic jsou nulové}
\bod{délková (a případně úhlová) {\bf jednotka}, která se v jednotlivých souřadnicích používá}
\vskip 4mm

Existuje mnoho různých souřadných systémů, kde každý má své praktické využití v určité, své výhody i nevýhody. Souřadné systémy je možné rozdělit podle několika kritérií. Prvním kritériem je {\it počet os}:

\vskip 4mm
\bod {Jedno-osé}
\bod {Dvou-osé}
\bod {Tří-osé}
\bod {Více-osé}
\vskip 4mm

Další možností jakým je možné rozdělit souřadné systémy je podle způsoby pozicování:

\vskip 4mm
\bod {Kartézský systém}
\bod {Polární systém}
\vskip 4mm

\Sekce{Dimenze souřadného systému}


\PodSekce{Jednorozměrná souřadnicová soustava}

Nejjednodušší používaná souřadnicová soustava je {\bf číselná osa}, jejíž počáteční bod je {\it bod nula}. Jedná se o jednorozměrný souřadnicový systém, který splňuje podmínku, že každému bodu udává vzdálenost a směr od jeho počátku - daný bod se nachází v určité vzdálenosti buď v kladném nebo záporném směru od počátku. Protože se jedná o jednorozměrný souřadnicový systém, poloha každého bodu na číselné ose je  určena pouze jednou reálnou hodnotou (kladnou nebo zápornou). V případě, že je daná číselná osa použita jako souřadnicová soustava popřípadě její část, pak je nazývána jako {\bf souřadnicová osa}.

\vskip 4mm
\centerline{ \pdfrefximage 1}
\vskip 4mm

\PodSekce {Dvourozměrná souřadnicová soustava}

Pro většinu praktických aplikací jednorozměrná souřadnicová soustava nedostačuje, protože nedokáže vyjádřit žádný jiný geometrický objekt než je bod (popřípadě vzdálenost mezi body - délku úsečky). Číselná osa ale slouží pro zavedení plošné souřadnicové soustavy, díky které lze popisovat (zakreslovat) body tvořící množinu nějakého plošného geometrického objektu (čtverec, trojúhelník, ...).

\Sekce{Kartézský souřadný systém}

Kartézská soustava souřadnic je takový souřadný systém, který je tvořen navzájem kolmými souřadnými (číselnými) osami, které se protínají v bodě, označovaném jako {\it počátek soustavy souřadnic O}, který je zároveň počátkem pro každou z těchto souřadných os. Měřítko se obvykle volí na všech osách stejně velký, nebo je v definici kartézského souřadného systému udán poměr v jakém jsou měřítka jednotlivých os zobrazovány V případě, že je jsou zvoleny různá měřítka na souřadných osách, je výsledný zobrazovaný útvar v kartézské soustavě deformován poměrem měřítek jednotlivých os. Jednotlivé hodnoty souřadnic mohou být i záporné (reálné). Absolutní hodnota dané souřadnice udává vzdálenost od počátku soustavy souřadnic na příslušné ose.

Jednotlivé souřadnice daného bodu tvoří uspořádanou n-tici (záleží na jejich pořadí pro správné propojení se souřadnými osami), kde {\it n} je počet dimenzí (os) kartézského systému, které jsou na základě množinové teorie zapisovány ve tvaru $[x,y,z, ...]$

Při kreslení grafu pomocí kartézského souřadného systému, je nutné dodržovat určitá pravidla, která pomáhají zpřehlednit výsledný zápis. Mezi tato pravidla patří:

\bod{Pojmenování os - osy x, y popřípadě z je třeba pojmenovat zapsáním daného písmenka k dané ose}
\bod{Vyznačit počátek - počátek se značí znakem O a nachází se v místě střetu všech souřadných os}
\bod{Vyznačit měřítko - měřítko se vyznačuje zapsáním souřadnice 1 na všech osách v kartézském souřadném systému}
\bod{Udání poměru měřítek - v případě, že na všech osách nejsou použity stejná měřítka, je nutné udat v jakém poměru jsou měřítka na jednotlivých osách}

\PodSekce{Kartézský souřadný systém v ploše}

Kartézský souřadný systém v ploše je tvořena dvěma navzájem kolmými souřadnými osami. Díky plošnému souřadnému systému je možné zobrazovat dvojrozměrné objekty jako je trojúhelník nebo čtverec. V {\it ploše} je vodorovná osa značena písmenem {\it x} a svislá osa písmenem {\it y}. Na těchto osách jsou vyznačovány x-ové a y-nové souřadnice jednotlivých bodů, přičemž je nutné brát v ohledu jednotky jednotlivých souřadných os.

Celá kartézská soustava souřadnic se dělí do čtyř částí - {\it kvadranty soustavy souřadnic}. Každý kvadrant představuje jednu čtvrtinu roviny souřadnic a je ohraničen příslušnými částmi souřadnicových os.

\vskip 4mm
\centerline{\pdfrefximage 3}
\vskip 4mm

\PodSekce {Kartézský souřadný systém v prostoru}

Kartézský souřadný systém v prostoru je rozšířením Kartézský souřadný systém v rovině o další rozměr (dimenzi), která je reprezentována souřadnou osou. Díky prostorovému souřadnému systému je možné zobrazit trojrozměrné objekty jako válec nebo kvádr. Kromě os x a y se v souřadném systému nachází ještě osa z, která udává výšku daného prostorového objektu. Jednotlivé body v prostoru jsou definovány pomocí tří čísel - souřadnic $[x,y,z]$

\vskip 4mm
\centerline{\pdfrefximage 4}
\vskip 4mm

\Sekce{Polární soustava souřadnic}

Polární soustava souřadnic je taková soustava souřadnic v rovině ve které jedna souřadnice udává vzdálenost od středu (poloměr r) souřadného systému a druhá souřadnice (označovaná $\varphi$) označuje úhel spojnice tohoto bodu a počátku souřadného systému od zvolené osy ležící v rovině (nejčastěji odpovídá ose {\it x} v kartézském souřadném systému). Každý bod je tedy stejně jako v kartézské soustavě souřadnic definovaný pomocí dvou čísel - vzdálenost bodu od počátku a úhel, který svírá spojnice bodu a počátku soustavy souřadnic. Úhly se v tomto případě měří dohodnutým způsobem proti směru hodinových ručiček. Pro označení vzdálenosti bodu se obvykle používá značka {\it r} a pro úhel písmeno řecké abecedy $\varphi$.

Polární souřadný systém se využívá v případech, kdy se nějaký bod pohybuje po kružnici a mění pouze vzdálenost (poloměr) tohoto pohybu od středu kružnice. Příkladem použití může být nějaký obráběcí stroj s radiálním pohybem, nebo jiné aplikace s krokovým motorem.

\vskip 4mm
\centerline{\pdfrefximage 2}
\vskip 4mm

\PodSekce{Převod mezi kartézskými a polárními souřadnicemi}

Souřadnice v polárním souřadném systému lze jednoduše převést na souřadnice kartézského souřadného systému a naopak. Bod je v polárním systému souřadnic určen vzdáleností od středu a úhlem, který svírá spojnice středu s bodem a kladnou poloosou x. Ty tvoří v kartézské soustavě souřadnic pravoúhlý trojúhelník, jehož strany lze popsat pomocí goniometrických funkcí {\it sinus} a {\it cosinus}.

\vskip 4mm
\centerline{\pdfrefximage 5}
\vskip 4mm

Na základě vztahů pravoúhlého trojúhelníku je možné převést kartézské souřadnice na polární pomocí vztahu:

$$ \varphi = atan({y \over x}) $$
$$ r = \sqrt{x^2+y^2} $$

Z polárních souřadnic do kartézských souřadnic je možné převést pomocí vztahu:

$$ x = cos(\varphi)\cdot r $$
$$ y = sin(\varphi)\cdot r $$

\Sekce {Shrnutí}

\Sekce{Vzdálenost dvou bodů v kartézské soustavě}





\end
